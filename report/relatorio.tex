\documentclass[12pt]{article}

\usepackage{sbc-template}
\usepackage[brazil,american]{babel}
\usepackage[utf8]{inputenc}

\usepackage{graphicx}
\usepackage{url}
\usepackage{float}
\usepackage{listings}
\usepackage{color}
\usepackage{todonotes}
\usepackage{algorithmic}
\usepackage{algorithm}
\usepackage{hyperref}
\usepackage{amssymb}
\usepackage{amsmath}
\graphicspath{{../parte1/graficos/}{../parte2/graficos/}}

\sloppy

\title{Trabalho 2\\
Ajuste de Curvas\\
\& \\
Splines}

\author{Dayanne Fernandes da Cunha, 13/0107191\\
       Yurick Hauschild, 12/0024136\\
       Grupo 6
}

\address{Dep. Matemática $-$ Universidade de Brasília (UnB)\\
  Cálculo Numérico $-$ Turma A
  \email{dayannefernandesc@gmail.com, yurick.hauschild@gmail.com}
}

\begin{document}
\maketitle

\selectlanguage{brazil}

 \begin{resumo}
 	Este relatório corresponde aos informativos das resoluções do Trabalho 2 de Cálculo Numérico da Turma A do semestre 2016/2.
 \end{resumo}

\section{Parte I : Ajuste de Curvas}
\label{sec:parte1}

\subsection{Questão 1}
\label{subsec:p1q1}

\textbf{Resolução:}


\subsection{Questão 2}
\label{subsec:p1q2}

\textbf{Resolução:}

\subsection{Questão 3}
\label{subsec:p1q3}

\textbf{Resolução:}

\subsection{Questão 4}
\label{subsec:p1q4}

\textbf{Resolução:} 

\section{Parte II : Splines}
\label{sec:parte2}

\subsection{Questão 5}
\label{subsec:p2q5}

\textbf{Resolução:}

\subsection{Questão 6}
\label{subsec:p2q6}

\textbf{Resolução:}

\subsection{Questão 7}
\label{subsec:p2q7}

\textbf{Resolução:}

\subsection{Questão 8}
\label{subsec:p2q8}

\textbf{Resolução:}

\subsection{Questão 9}
\label{subsec:p2q9}

\textbf{Resolução:} 

\end{document}
