\documentclass[12pt]{article}

\usepackage{sbc-template}
\usepackage[brazil,american]{babel}
\usepackage[utf8]{inputenc}

\usepackage{graphicx}
\usepackage{url}
\usepackage{float}
\usepackage{listings}
\usepackage{color}
\usepackage{todonotes}
\usepackage{algorithmic}
\usepackage{algorithm}
\usepackage{hyperref}
\usepackage{amssymb}
\usepackage{amsmath}
\usepackage{enumitem}
\graphicspath{{../parte1/graficos/}{../parte2/graficos/}}

\sloppy

\title{Trabalho 2\\
Ajuste de Curvas\\
\& \\
Splines}

\author{Dayanne Fernandes da Cunha, 13/0107191\\
       Yurick Hauschild, 12/0024136\\
       Grupo 6
}

\address{Dep. Matemática $-$ Universidade de Brasília (UnB)\\
  Cálculo Numérico $-$ Turma A
  \email{dayannefernandesc@gmail.com, yurick.hauschild@gmail.com}
}

\begin{document}
\maketitle

\selectlanguage{brazil}

 \begin{resumo}
 	Este relatório corresponde aos informativos das resoluções do Trabalho 2 de Cálculo Numérico da Turma A do semestre 2016/2.
 \end{resumo}

\section{Parte I : Ajuste de Curvas}
\label{sec:parte1}

Esta primeira parte de trabalho será dedicada aos ajustes de curvas pelo método dos mínimos quadrados. Para tal, considere o conjunto de dados apresentado na Figura~\ref{fig:tab1}.

\begin{figure}[H]
	\centering
	\includegraphics[width=0.3\textwidth]{tab1.png}
	\caption{Tabela de pontos.}
	\label{fig:tab1}
\end{figure}

\subsection{Questão 1}
\label{subsec:p1q1}

Determine, para o conjunto de pontos da Figura~\ref{fig:tab1}, os melhores ajustes polinomiais de grau 1, 2, 3, 5 e 10. Trace os seus resultados e comente sobre a adequação de cada um destes ajustes. É possível encontrar uma maneira quantitativa de se julgar qual dentre estes cinco ajustes é o melhor?

\textbf{Resolução:}

O método dos quadrados mínimos foi implementado e pode ser encontrado no arquivo "$\textit{parte1/q1.f90}$".

A partir dele, foi possível encontrar os seguintes ajustes:

\begin{eqnarray}
    \begin{split}
        y_{1} = 5.4491559 - 1.819038x \\
        y_{2} = 8.7060667 - 12.10402x + 5.7138785x^{2} \\
        y_{3} = 11.099895 - 26.10301x + 24.612525x^{2} - 6.999498x^{3} \\
        y_{5} = 11.820173 - 30.99526x + 29.737760x^{2} - 0.489529x^{3} - 10.879890x^{4} + 3.668555x^{5} \\
        y_{10} = 10.767245 - 21.21473x + 8.5667448x^{2} + 7.315704x^{3} + 0.6093679x^{4} - 1.894178x^{5} - 1.310241x^{6} - 0.297909x^{7} + 0.1513477x^{8} + 0.159235x^{9} + 0.025067x^{10}
    \end{split}
\end{eqnarray}

Cada um deles está representado graficamente e pode ser encontrado nos arquivos "$parte1/graficos/q1.1.png$", "$parte1/graficos/q1.2.png$", "$parte1/graficos/q1.3.png$", "$parte1/graficos/q1.5.png$" e "$parte1/graficos/q1.10.png$", respectivamente.
É fácil de verificar pelos gráficos que nenhum dos ajustes realmente cobriu perfeitamente os pontos da tabela, mas dada a grande discrepância entre vários valores muitas vezes consecutivos, isto é um comportamento bastante compreensível.

Utilizando-se do coeficiente de determinação $R^{2}$, é possível mensurar a adequação de cada um destes ajustes e, por conseguinte, decidir qual dentre eles é o melhor.

O coeficiente $R^{2}$ é calculado a partir da fórmula

\begin{equation}
    R^{2} = 1 - \frac{SQ_{err}}{SQ_{tot}}
\end{equation}

Onde $SQ_{err}$ é a soma dos quadrados dos erros e ${SQ_{tot}}$ é a soma dos quadrados total.

Temos, então, o resultado:

\begin{itemize}[noitemsep]
\item 71,9055769\% da variância de $y_{1}$ é explicada pela tabela;
\item 78,8906477\% da variância de $y_{2}$ é explicada pela tabela;
\item 80,7774013\% da variância de $y_{3}$ é explicada pela tabela;
\item 81,0324401\% da variância de $y_{5}$ é explicada pela tabela;
\item 81,3383318\% da variância de $y_{10}$ é explicada pela tabela.
\end{itemize}

Concluímos, então, que $y_{10}$, dentre os ajustes calculados, é o que melhor aproxima a função proposta na tabela.
Por outro lado, é notável que a melhora entre $y_{5}$ e $y_{10}$ foi mínima.

\subsection{Questão 2}
\label{subsec:p1q2}

\textbf{Resolução:}

\subsection{Questão 3}
\label{subsec:p1q3}

\textbf{Resolução:}

\subsection{Questão 4}
\label{subsec:p1q4}

\textbf{Resolução:}

\section{Parte II : Splines}
\label{sec:parte2}

\subsection{Questão 5}
\label{subsec:p2q5}

\textbf{Resolução:}

\subsection{Questão 6}
\label{subsec:p2q6}

\textbf{Resolução:}

\subsection{Questão 7}
\label{subsec:p2q7}

\textbf{Resolução:}

\subsection{Questão 8}
\label{subsec:p2q8}

\textbf{Resolução:}

\subsection{Questão 9}
\label{subsec:p2q9}

\textbf{Resolução:}

\end{document}
